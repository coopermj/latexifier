\documentclass[11pt,letterpaper]{article}
\usepackage[margin=1in]{geometry}
\usepackage{xcolor}
\usepackage{listings}
\usepackage{hyperref}
\usepackage{fancyhdr}
\usepackage{titlesec}
\usepackage{enumitem}
\usepackage{tcolorbox}
\usepackage{fontawesome5}
\usepackage{amssymb}

% Colors
\definecolor{primary}{HTML}{2563EB}
\definecolor{codebg}{HTML}{F3F4F6}
\definecolor{codeframe}{HTML}{D1D5DB}
\definecolor{success}{HTML}{059669}
\definecolor{warning}{HTML}{D97706}

% Hyperlinks
\hypersetup{
    colorlinks=true,
    linkcolor=primary,
    urlcolor=primary
}

% Code listings
\lstset{
    basicstyle=\ttfamily\small,
    backgroundcolor=\color{codebg},
    frame=single,
    rulecolor=\color{codeframe},
    breaklines=true,
    breakatwhitespace=true,
    postbreak=\mbox{\textcolor{red}{$\hookrightarrow$}\space},
    showstringspaces=false,
    tabsize=2,
    xleftmargin=0.5em,
    xrightmargin=0.5em,
    aboveskip=1em,
    belowskip=1em
}

% Define JSON language for listings
\lstdefinelanguage{json}{
    basicstyle=\ttfamily\small,
    morestring=[b]",
    literate=
     *{:}{{{\color{primary}{:}}}}{1}
      {,}{{{\color{primary}{,}}}}{1}
      {\{}{{{\color{primary}{\{}}}}{1}
      {\}}{{{\color{primary}{\}}}}}{1}
      {[}{{{\color{primary}{[}}}}{1}
      {]}{{{\color{primary}{]}}}}{1}
}

% Section formatting
\titleformat{\section}{\Large\bfseries\color{primary}}{\thesection}{1em}{}
\titleformat{\subsection}{\large\bfseries}{\thesubsection}{1em}{}

% Header/Footer
\pagestyle{fancy}
\fancyhf{}
\fancyhead[L]{\textcolor{primary}{\textbf{Latexifier}}}
\fancyhead[R]{\textcolor{gray}{API Documentation}}
\fancyfoot[C]{\thepage}
\renewcommand{\headrulewidth}{0.4pt}

% Custom boxes
\newtcolorbox{infobox}{
    colback=blue!5,
    colframe=primary,
    boxrule=0.5pt,
    left=1em,
    right=1em,
    top=0.5em,
    bottom=0.5em
}

\newtcolorbox{warningbox}{
    colback=yellow!10,
    colframe=warning,
    boxrule=0.5pt,
    left=1em,
    right=1em,
    top=0.5em,
    bottom=0.5em
}

\begin{document}

\begin{center}
    {\Huge\bfseries\color{primary} Latexifier}\\[0.5em]
    {\Large LaTeX-to-PDF Web Service}\\[1em]
    {\large API Documentation \& Usage Guide}\\[0.5em]
    {\color{gray}\today}
\end{center}

\vspace{2em}

\begin{infobox}
\textbf{Service URL:} \url{https://latexifier-production.up.railway.app}\\
\textbf{Authentication:} API Key via \texttt{X-API-Key} header
\end{infobox}

\tableofcontents
\newpage

%==============================================================================
\section{Overview}
%==============================================================================

Latexifier is a web service that compiles LaTeX documents to PDF. It supports:

\begin{itemize}[leftmargin=2em]
    \item \textbf{Multiple TeX engines:} pdflatex, xelatex, lualatex
    \item \textbf{Flexible input:} Single files, multi-file projects, ZIP archives
    \item \textbf{Custom assets:} Upload and persist fonts and style files
    \item \textbf{Package management:} Install additional TeX packages on demand
    \item \textbf{API-first design:} OpenAPI spec for easy integration
\end{itemize}

%==============================================================================
\section{API Endpoints}
%==============================================================================

\begin{tabular}{|l|l|p{6cm}|}
\hline
\textbf{Method} & \textbf{Endpoint} & \textbf{Description} \\
\hline
POST & \texttt{/compile} & Compile LaTeX to PDF \\
GET & \texttt{/health} & Health check and version info \\
POST & \texttt{/styles} & Upload a style file (.sty, .cls) \\
GET & \texttt{/styles} & List uploaded styles \\
DELETE & \texttt{/styles/\{name\}} & Delete a style \\
POST & \texttt{/fonts} & Upload a font file \\
GET & \texttt{/fonts} & List uploaded fonts \\
DELETE & \texttt{/fonts/\{name\}} & Delete a font \\
POST & \texttt{/packages/install} & Install TeX packages \\
GET & \texttt{/packages} & List installed packages \\
GET & \texttt{/openapi.json} & OpenAPI specification \\
\hline
\end{tabular}

%==============================================================================
\section{Using curl}
%==============================================================================

\subsection{Health Check}

\begin{lstlisting}[language=bash]
curl https://latexifier-production.up.railway.app/health
\end{lstlisting}

\subsection{Single File Compilation}

\begin{lstlisting}[language=bash]
# Create a simple LaTeX file
echo '\documentclass{article}
\begin{document}
Hello World!
\end{document}' > doc.tex

# Base64 encode and compile
curl -X POST \
  https://latexifier-production.up.railway.app/compile \
  -H "Content-Type: application/json" \
  -H "X-API-Key: YOUR_API_KEY" \
  -d "{
    \"content\": \"$(base64 -i doc.tex)\",
    \"filename\": \"doc.tex\",
    \"output_format\": \"base64\"
  }" | jq -r '.pdf' | base64 -d > output.pdf
\end{lstlisting}

\subsection{Multi-File Compilation}

For documents with images or multiple source files:

\begin{lstlisting}[language=bash]
curl -X POST \
  https://latexifier-production.up.railway.app/compile \
  -H "Content-Type: application/json" \
  -H "X-API-Key: YOUR_API_KEY" \
  -d "{
    \"files\": [
      {
        \"name\": \"main.tex\",
        \"content\": \"$(base64 -i main.tex)\"
      },
      {
        \"name\": \"image.png\",
        \"content\": \"$(base64 -i image.png)\"
      }
    ],
    \"main_file\": \"main.tex\"
  }"
\end{lstlisting}

\subsection{ZIP Archive Compilation}

\begin{lstlisting}[language=bash]
# Create a ZIP with your project
zip -r project.zip main.tex images/ chapters/

# Submit the ZIP
curl -X POST \
  https://latexifier-production.up.railway.app/compile \
  -H "Content-Type: application/json" \
  -H "X-API-Key: YOUR_API_KEY" \
  -d "{
    \"zip\": \"$(base64 -i project.zip)\",
    \"main_file\": \"main.tex\"
  }"
\end{lstlisting}

%==============================================================================
\section{Using Python}
%==============================================================================

\subsection{Basic Compilation}

\begin{lstlisting}[language=python]
import requests
import base64

API_URL = "https://latexifier-production.up.railway.app"
API_KEY = "YOUR_API_KEY"

# Read and encode LaTeX file
with open("document.tex", "rb") as f:
    content = base64.b64encode(f.read()).decode()

# Compile
response = requests.post(
    f"{API_URL}/compile",
    headers={"X-API-Key": API_KEY},
    json={
        "content": content,
        "filename": "document.tex",
        "output_format": "base64"
    }
)

result = response.json()
if result["success"]:
    pdf_bytes = base64.b64decode(result["pdf"])
    with open("output.pdf", "wb") as f:
        f.write(pdf_bytes)
    print("PDF saved to output.pdf")
else:
    print(f"Error: {result['error']}")
    print(f"Log: {result['log']}")
\end{lstlisting}

\subsection{Multi-File with Images}

\begin{lstlisting}[language=python]
import requests
import base64

def encode_file(path):
    with open(path, "rb") as f:
        return base64.b64encode(f.read()).decode()

response = requests.post(
    f"{API_URL}/compile",
    headers={"X-API-Key": API_KEY},
    json={
        "files": [
            {"name": "main.tex", "content": encode_file("main.tex")},
            {"name": "fig1.png", "content": encode_file("fig1.png")},
            {"name": "fig2.pdf", "content": encode_file("fig2.pdf")}
        ],
        "main_file": "main.tex",
        "engine": "pdflatex"
    }
)
\end{lstlisting}

\subsection{Using XeLaTeX for Custom Fonts}

\begin{lstlisting}[language=python]
response = requests.post(
    f"{API_URL}/compile",
    headers={"X-API-Key": API_KEY},
    json={
        "content": content,
        "filename": "document.tex",
        "engine": "xelatex"  # Required for fontspec
    }
)
\end{lstlisting}

%==============================================================================
\section{TeX Engine Selection}
%==============================================================================

\begin{tabular}{|l|p{9cm}|}
\hline
\textbf{Engine} & \textbf{Use Case} \\
\hline
\texttt{pdflatex} & Default. Standard LaTeX with good compatibility. Fast compilation. \\
\texttt{xelatex} & Custom fonts via \texttt{fontspec}, full Unicode support, simpler font setup. \\
\texttt{lualatex} & Lua scripting, custom fonts, advanced typography, programmable documents. \\
\hline
\end{tabular}

\vspace{1em}

\begin{warningbox}
\textbf{Note:} If using custom fonts with \texttt{fontspec}, you must use \texttt{xelatex} or \texttt{lualatex}. The \texttt{pdflatex} engine does not support \texttt{fontspec}.
\end{warningbox}

\subsection{When to Use Each Engine}

\textbf{Use pdflatex when:}
\begin{itemize}[leftmargin=2em]
    \item You need maximum compatibility with older packages
    \item Speed is critical (fastest compilation)
    \item You don't need custom system fonts
    \item Your document uses only ASCII or standard encodings
\end{itemize}

\textbf{Use xelatex when:}
\begin{itemize}[leftmargin=2em]
    \item You need custom TrueType/OpenType fonts
    \item Your document contains Unicode characters (CJK, Arabic, etc.)
    \item You want simple font setup without Lua complexity
    \item You're migrating from pdflatex and need fontspec
\end{itemize}

\textbf{Use lualatex when:}
\begin{itemize}[leftmargin=2em]
    \item You need programmatic document generation with Lua scripts
    \item You want advanced microtypography (\texttt{microtype} works better)
    \item You need to manipulate the PDF directly
    \item You're using packages that require Lua (e.g., \texttt{luacode}, \texttt{selnolig})
    \item You need advanced font features like OpenType math fonts
\end{itemize}

\subsection{XeLaTeX vs LuaLaTeX Comparison}

\begin{tabular}{|l|c|c|}
\hline
\textbf{Feature} & \textbf{XeLaTeX} & \textbf{LuaLaTeX} \\
\hline
Custom fonts (fontspec) & \checkmark & \checkmark \\
Unicode support & \checkmark & \checkmark \\
Compilation speed & Faster & Slower \\
Lua scripting & -- & \checkmark \\
Microtype support & Limited & Full \\
Direct PDF manipulation & -- & \checkmark \\
OpenType math & \checkmark & \checkmark \\
Memory for large docs & Limited & Better \\
\hline
\end{tabular}

%==============================================================================
\section{LuaLaTeX Deep Dive}
%==============================================================================

LuaLaTeX embeds the Lua programming language directly into the TeX engine, enabling powerful programmatic document generation.

\subsection{Basic LuaLaTeX Document}

\begin{lstlisting}[language=tex]
\documentclass{article}
\usepackage{fontspec}
\usepackage{luacode}

\setmainfont{TeX Gyre Termes}

\begin{document}
\section{Hello from LuaLaTeX}

% Inline Lua
The answer is \directlua{tex.print(6 * 7)}.

\end{document}
\end{lstlisting}

\subsection{Using Lua for Dynamic Content}

\begin{lstlisting}[language=tex]
\documentclass{article}
\usepackage{luacode}

\begin{luacode*}
function fibonacci(n)
    if n <= 1 then return n end
    return fibonacci(n-1) + fibonacci(n-2)
end

function printFibSequence(count)
    local result = {}
    for i = 0, count-1 do
        table.insert(result, fibonacci(i))
    end
    tex.print(table.concat(result, ", "))
end
\end{luacode*}

\newcommand{\fibsequence}[1]{\directlua{printFibSequence(#1)}}

\begin{document}
First 10 Fibonacci numbers: \fibsequence{10}
\end{document}
\end{lstlisting}

\subsection{Reading External Data with Lua}

\begin{lstlisting}[language=tex]
\documentclass{article}
\usepackage{luacode}

\begin{luacode*}
function processCSV(filename)
    local file = io.open(filename, "r")
    if not file then
        tex.print("File not found")
        return
    end

    tex.print("\\begin{tabular}{|l|l|}")
    tex.print("\\hline")

    for line in file:lines() do
        local cols = {}
        for col in line:gmatch("[^,]+") do
            table.insert(cols, col)
        end
        tex.print(table.concat(cols, " & ") .. " \\\\")
        tex.print("\\hline")
    end

    tex.print("\\end{tabular}")
    file:close()
end
\end{luacode*}

\newcommand{\csvtable}[1]{\directlua{processCSV("#1")}}

\begin{document}
\csvtable{data.csv}
\end{document}
\end{lstlisting}

\subsection{Advanced Font Features with LuaLaTeX}

\begin{lstlisting}[language=tex]
\documentclass{article}
\usepackage{fontspec}
\usepackage{unicode-math}  % OpenType math fonts

% Main text font
\setmainfont{TeX Gyre Pagella}[
    Numbers=OldStyle,
    Ligatures=TeX
]

% Math font
\setmathfont{TeX Gyre Pagella Math}

% Monospace with specific features
\setmonofont{Fira Code}[
    Contextuals=Alternate,  % Programming ligatures
    Scale=0.9
]

\begin{document}
Math example: $\int_0^\infty e^{-x^2} dx = \frac{\sqrt{\pi}}{2}$

Old-style figures: 0123456789

Code: \texttt{=> != <= >=}
\end{document}
\end{lstlisting}

\subsection{Using LuaLaTeX via the API}

\begin{lstlisting}[language=python]
import requests
import base64

API_URL = "https://latexifier-production.up.railway.app"
API_KEY = "YOUR_API_KEY"

lua_document = r'''
\documentclass{article}
\usepackage{luacode}

\begin{luacode*}
function greet(name)
    tex.print("Hello, " .. name .. "!")
end
\end{luacode*}

\newcommand{\greet}[1]{\directlua{greet("#1")}}

\begin{document}
\greet{World}

Today's random number: \directlua{tex.print(math.random(1,100))}
\end{document}
'''

content = base64.b64encode(lua_document.encode()).decode()

response = requests.post(
    f"{API_URL}/compile",
    headers={"X-API-Key": API_KEY},
    json={
        "content": content,
        "filename": "document.tex",
        "engine": "lualatex",  # Must specify lualatex!
        "output_format": "base64"
    }
)

result = response.json()
if result["success"]:
    with open("output.pdf", "wb") as f:
        f.write(base64.b64decode(result["pdf"]))
\end{lstlisting}

\subsection{LuaLaTeX-Specific Packages}

\begin{tabular}{|l|p{8cm}|}
\hline
\textbf{Package} & \textbf{Purpose} \\
\hline
\texttt{luacode} & Clean Lua code blocks in LaTeX \\
\texttt{luatexbase} & Base utilities for LuaTeX programming \\
\texttt{unicode-math} & OpenType math font support \\
\texttt{selnolig} & Intelligent ligature suppression \\
\texttt{luaotfload} & OpenType font loading (auto-loaded by fontspec) \\
\texttt{luamplib} & MetaPost graphics in LuaLaTeX \\
\texttt{chickenize} & Fun text transformations (demonstrates Lua callbacks) \\
\texttt{lua-visual-debug} & Visual debugging of boxes and glue \\
\hline
\end{tabular}

\begin{infobox}
\textbf{Performance Tip:} LuaLaTeX is slower than pdflatex and xelatex due to Lua overhead. For simple documents without Lua scripting, prefer xelatex for custom fonts.
\end{infobox}

%==============================================================================
\section{Managing Fonts}
%==============================================================================

\subsection{Uploading Fonts}

Fonts persist across deployments when stored on the Railway volume.

\begin{lstlisting}[language=bash]
# Upload a font
curl -X POST \
  https://latexifier-production.up.railway.app/fonts \
  -H "X-API-Key: YOUR_API_KEY" \
  -F "file=@MyFont-Regular.ttf"
\end{lstlisting}

\subsection{Using Custom Fonts in LaTeX}

\begin{lstlisting}[language=tex]
\documentclass{article}
\usepackage{fontspec}
\setmainfont{MyFont-Regular.ttf}
\begin{document}
This text uses my custom font.
\end{document}
\end{lstlisting}

\begin{warningbox}
\textbf{Font Licensing:} Only upload fonts you have license to use on a server. Adobe Fonts and other subscription fonts typically prohibit server installation.
\end{warningbox}

\subsection{Free Font Alternatives}

\begin{itemize}
    \item \textbf{Google Fonts:} \url{https://fonts.google.com} (OFL license)
    \item \textbf{Font Squirrel:} \url{https://fontsquirrel.com}
    \item \textbf{The League of Moveable Type:} \url{https://theleagueofmoveabletype.com}
\end{itemize}

%==============================================================================
\section{Managing Styles}
%==============================================================================

Upload custom \texttt{.sty} or \texttt{.cls} files:

\begin{lstlisting}[language=bash]
# Upload a style file
curl -X POST \
  https://latexifier-production.up.railway.app/styles \
  -H "X-API-Key: YOUR_API_KEY" \
  -F "file=@mystyle.sty"

# List styles
curl https://latexifier-production.up.railway.app/styles \
  -H "X-API-Key: YOUR_API_KEY"
\end{lstlisting}

%==============================================================================
\section{Installing TeX Packages}
%==============================================================================

Install additional packages from CTAN:

\begin{lstlisting}[language=bash]
# Install a package
curl -X POST \
  https://latexifier-production.up.railway.app/packages/install \
  -H "Content-Type: application/json" \
  -H "X-API-Key: YOUR_API_KEY" \
  -d '{"packages": ["tikz-cd", "minted"]}'

# List installed packages
curl https://latexifier-production.up.railway.app/packages \
  -H "X-API-Key: YOUR_API_KEY"
\end{lstlisting}

\begin{infobox}
\textbf{Note:} The service uses TeX Live Full, so most common packages are already installed.
\end{infobox}

%==============================================================================
\section{ChatGPT GPT Action Setup}
%==============================================================================

\subsection{Step 1: Create a Custom GPT}

In ChatGPT, go to \textbf{Explore GPTs} $\rightarrow$ \textbf{Create}.

\subsection{Step 2: Add an Action}

\begin{enumerate}
    \item Go to \textbf{Configure} $\rightarrow$ \textbf{Actions} $\rightarrow$ \textbf{Create new action}
    \item Click \textbf{Import from URL}
    \item Enter: \texttt{https://latexifier-production.up.railway.app/openapi.json}
\end{enumerate}

\subsection{Step 3: Configure Authentication}

\begin{itemize}
    \item \textbf{Authentication Type:} API Key
    \item \textbf{Auth Type:} Custom Header
    \item \textbf{Header Name:} \texttt{X-API-Key}
    \item \textbf{API Key:} Your API key
\end{itemize}

\subsection{Step 4: Add Instructions}

Add to your GPT's instructions:

\begin{lstlisting}
When asked to create a LaTeX document:

1. Generate the LaTeX source code
2. Base64-encode the content
3. Call the compile action with:
   - content: base64-encoded LaTeX
   - filename: "document.tex"
   - engine: "pdflatex" (or "xelatex" for custom fonts)
   - output_format: "base64"

4. The response contains a base64-encoded PDF.

For multi-file documents with images:
- Use the "files" parameter with an array of
  {name, content} objects
- Set "main_file" to the primary .tex file
\end{lstlisting}

%==============================================================================
\section{Error Handling}
%==============================================================================

\subsection{Response Format}

All \texttt{/compile} responses include:

\begin{lstlisting}[language=json]
{
  "success": true,
  "pdf": "base64-encoded-pdf...",
  "error": null,
  "log": "LaTeX compilation log..."
}
\end{lstlisting}

On failure:

\begin{lstlisting}[language=json]
{
  "success": false,
  "pdf": null,
  "error": "LaTeX compilation failed (run 1)",
  "log": "! Undefined control sequence..."
}
\end{lstlisting}

\subsection{Common Errors}

\begin{tabular}{|l|p{8cm}|}
\hline
\textbf{Error} & \textbf{Solution} \\
\hline
Missing package & Install via \texttt{/packages/install} or add to ZIP \\
Undefined control sequence & Check for typos or missing \texttt{\textbackslash usepackage} \\
Font not found & Upload font via \texttt{/fonts} and use xelatex \\
File not found & Ensure all referenced files are in the files array \\
Timeout (120s) & Simplify document or split into parts \\
\hline
\end{tabular}

%==============================================================================
\section{Best Practices}
%==============================================================================

\begin{enumerate}
    \item \textbf{Use pdflatex} unless you need custom fonts or Unicode
    \item \textbf{Include all dependencies} in your files array or ZIP
    \item \textbf{Check the log} on failures---it contains detailed error info
    \item \textbf{Use relative paths} for images: \texttt{\textbackslash includegraphics\{image.png\}}
    \item \textbf{Test locally first} before submitting to the API
\end{enumerate}

%==============================================================================
\section{Quick Reference}
%==============================================================================

\begin{infobox}
\textbf{Base URL:} \texttt{https://latexifier-production.up.railway.app}

\textbf{Headers:}
\begin{itemize}[leftmargin=1em,topsep=0pt]
    \item \texttt{Content-Type: application/json}
    \item \texttt{X-API-Key: YOUR\_API\_KEY}
\end{itemize}

\textbf{Compile Request Fields:}
\begin{itemize}[leftmargin=1em,topsep=0pt]
    \item \texttt{content} -- Base64 LaTeX (single file)
    \item \texttt{files} -- Array of \{name, content\} (multi-file)
    \item \texttt{zip} -- Base64 ZIP archive
    \item \texttt{main\_file} -- Entry point (default: main.tex)
    \item \texttt{engine} -- pdflatex, xelatex, lualatex
    \item \texttt{output\_format} -- pdf, base64
\end{itemize}
\end{infobox}

\vfill
\begin{center}
\textcolor{gray}{\small Generated by Latexifier $\bullet$ \url{https://latexifier-production.up.railway.app}}
\end{center}

\end{document}
